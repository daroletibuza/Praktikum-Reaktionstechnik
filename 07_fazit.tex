\pagebreak
\section{Fazit}
\label{sec:fazit}

Der berechnete Stoßfaktor $k_0$ beträgt $1,6*10^6$ und die Aktivierungsenergie \SI{59,6}{\kilo\joule\per\mole}. Die Reaktionordnung beläuft sich auf n $\approx2,2$.  (vgl. Tab.\ref{tab:kin_kon})\\


Für die Rührkesselkaskade aus drei Kesseln konnten die Kenngrößen Bodensteinzal und Kesselzahl zwar ermittelt werden, die erhaltenen Ergebnisse unterscheiden sich allerdings stark voneinander. Je nach dem ob eine Impuls- oder eine Verdrängungsmarkierung durchgeführt wurde. Die Bodensteinzahl der Rührkesselkaskade bewegt sich demnach zwischen 2,6 und 8,9. Die Kesselzahl liegt zwischen 2,4 und 5,5.\\ (vgl. Tab.\ref{tab:vwz_impuls} und Tab.\ref{tab:vwz_verdrang})\\

Der Umsatzgrad der Reaktion ist von der Reaktionstemperatur abhängig. Allgemein gilt, dass er mit mit der Temperatur ansteigt. Der, nach Durchlauf von drei Rührkesseln, erzielte Umsatzgrad lag bei 73\% (vgl. Tab.\ref{tab:usg_experimentell}). Dies liegt unter dem erwarteten und zuvor anlalytisch bestimmten Umsatzgrad von 77\% (vgl. Tab.\ref{tab:usg_analytisch}). \\

Die Berechnung eines kontinuierlich betriebenen Reaktors bei einer Temperatur von \SI{32}{\degreeCelsius} und einem Umsatz von \SI{100000}{\tonne\per\year} ergab ein Reaktorvolumen von rund \SI{215}{\cubic\meter}. Das Ergebnis ist nur zur Veranschaulichung der Größenordnung geeignet. Das Reaktorvolumen ließe sich durch eine Erhöhung der Reaktionstemperatur und der damit einhergehenden Verweilzeitverkürzung verkleinern.\\ 

Die berechneten Kennzahlen und Umsatzgrade sind sehr kritisch zu handhaben. Die Verwendeten Formeln sind größtenteils empirischen Ursprungs und keine exakt herleitbaren Gesetzmäßigkeiten. Erschwerend kommt hinzu, dass Messfehlerbehaftete Größen potenziert in den Rechnungen vorkommen. Kleinste Fehler haben dadurch deutlich spürbare Auswirkungen. Es ist von Abweichungen von 10\% und mehr auszugehen.
 



