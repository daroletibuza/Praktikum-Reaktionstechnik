\section{Auswertung}
\label{sec:auswertung}

%Darstellung Messungen und Kurzbeschreibung
\subsection{Kinetik}
\label{subsec:kinetik}
Aus den gemessenen Leitfähigkeiten des Versuchs "`Kinetik"' wurde der Umsatzgrad X(t) bestimmt \mbox{(siehe Gl. \ref{gl:1})}. Die Leitfähigkeit $\lambda_\infty$ wurde zum Ende des Praktikums, nach vollständig abgelaufener Reaktion gemessen.

\begin{flalign}
\label{gl:1}
	X(t) &= \frac{|\lambda_0-\lambda(t)|}{|\lambda_0-\lambda_\infty|}
\end{flalign}

Mit dem Umsatzgrad X(t) und der Ausgangskonzentration $c_0(NaOH)$ lässt sich der Konzentrationsverlauf c(t) von NaOH ermitteln \mbox{(siehe Gl. \ref{gl:2})} und wie in Abb. \ref{dia:c/t} für die drei Versuchstemperaturen darstellen.
\begin{flalign}
\label{gl:2}
c(t) &= (1-X(t))*c_0(\text{NaOH})
\end{flalign}

\begin{figure}[h!]
	\begin{center}
		\resizebox{0.8\textwidth}{!}{
		\begin{tikzpicture}[trim axis left, trim axis right]
		\begin{axis}[
		axis lines = left,
		width = 13cm,
		height = 7cm,
		xmin = 0,
		xmax = 900,
		ymin = 0,
		ymax = 55,
		%ytick = {0,2,...,14},
		%xtick = {0,10,...,100},
		ylabel={c(NaOH) in [\si{\milli\mole\per\liter}]},
		xlabel={t in [\si{\second}]},
	%	legend style={at={(0.5,0.95)},anchor=west}
		]
		\addplot table {Konzentration20.dat};
		\addplot table {Konzentration30.dat};
		\addplot table {Konzentration40.dat};
		
		\legend{T=\SI{23,5}{\celsius},T=\SI{30,2}{\celsius},T=\SI{37,0}{\celsius}};
		\end{axis}
		\end{tikzpicture}
			}
		\caption{Konzentration von NaOH über der Zeit}
		\label{dia:c/t}
	\end{center}
\end{figure}
\FloatBarrier

Aus den Stöchiometriefaktoren der zugrundeliegenden Reaktionsgleichung und dem Konzentrationsverlauf lässt sich des Weiteren die Reaktionsgeschwindigkeit $r$ bestimmen.

\begin{center}
	\ce{CH3COOCH2CH3 + NaOH -> CH3COO- + Na+ + C2H5OH}
\end{center}

Für die Natronlauge als Edukt ergibt sich somit ein Stöchiometriekoeffizient von $\nu(NaOH)= -1$. Kombiniert werden diese Zusammenhänge in Gl. \ref{gl:3} mit Hilfe der Stoffmengenänderungsgeschwindigkeit $R$.
\begin{flalign}
\label{gl:3}
	R_{\text{NaOH}} &= \nu(NaOH) * r \approx \frac{\Delta c(NaOH)}{\Delta t}\\
	r				&= \nu(NaOH) * \frac{\Delta c(NaOH)}{\Delta t}
\end{flalign}

Um über die Reaktionsgeschwindigkeit den Stoßfaktor $k$ bestimmen zu können, wird diese in Gl. \ref{gl:4}, mit dem Potenzansatz in Zusammenhang gebracht.
In Abb.\ref{dia:lnr/lnc} sind die linearen Verläufe der Messwerte unter Einbeziehung von Gl. \ref{gl:4} für die untersuchten Temperaturen aufgetragen.

\begin{flalign}
\label{gl:4}
-r	&= -k*c(NaOH)^n\\
 r	&= k*c(NaOH)^n\\
 \ln{r} &= \ln{k} + \ln{c(NaOH)}*n
\end{flalign}

\begin{figure}[h!]
	\begin{center}
		%\resizebox{0.8\textwidth}{!}{
		\begin{tikzpicture}[trim axis left, trim axis right]
		\begin{axis}[
		axis lines = left,
		width = 13cm,
		height = 7cm,
		xmin = 2.8,
		xmax = 3.9,
		ymin = -4.5,
		ymax = -0.5,
		ytick = {-4.5,-4,...,-1},
		xtick = {2.8,2.9,...,3.9},
		ylabel={ln(r)},
		y label style={at={(-0.03,0.5)}},
		xlabel={ln(c)},
		legend style={at={(0.95,0.95)},anchor=west}
		]
		\addplot table {lnclnr20grad.dat};
		\addplot table {lnclnr30grad.dat};
		\addplot table {lnclnr40grad.dat};
		\addplot +[mark=none, dashed, black] {2.14*x - 8.83};
		\addplot +[mark=none, dashed, black] {2.1925*x - 9.3751};
		\addplot +[mark=none, dashed, black] {2.21*x - 9.88};
	
		\legend{T=\SI{23,5}{\celsius},T=\SI{30,2}{\celsius},T=\SI{37,0}{\celsius}};
		\end{axis}
		\end{tikzpicture}
		%	}Ventilkennlinie
		\caption{Logarithmus der Reaktionsgeschwindigkeit über dem Logarithmus der Konzentration (in diesem Versuch die Leitfähigkeit nach dem Ende des Praktikums)}
		\label{dia:lnr/lnc}
	\end{center}
\end{figure}
\FloatBarrier
\vspace*{-5mm}

\textbf{Regressionsgeraden $y$ und Bestimmtheitsmaß $R^2$:}
\begin{flalign}
	y(T=\SI{23,5}{\celsius}) &= 2,21*x - 9,88 \quad | \quad R^2 = 0,8911\\
	y(T=\SI{32,2}{\celsius}) &= 2,19*x - 9,38 \quad | \quad R^2 = 0,9016\\
	y(T=\SI{37,0}{\celsius}) &= 2,14*x - 8,83 \quad | \quad R^2 = 0,9179 \nonumber
\end{flalign}
\vspace*{-10mm}

%Tabelle START
\renewcommand{\arraystretch}{1.2}
\begin{table}[h!]
	\centering
	\caption{Kinetische Kennwerte}
	\label{tab:kin_kenn}
	\makebox[\textwidth]{
		\resizebox{18cm}{!}{
			\begin{tabular}{c|c|c|c|c}
				\hline
				\textbf{Temperatur in \si{\celsius}} & \textbf{$\ln{\text{k}}$} & \textbf{Geschwindigkeitskonstante k in \si{\liter\per\milli \mole \per \second}} & \textbf{$\frac{1}{\text{T}}$ in \si{\kelvin}}&  \textbf{Reaktionsordnung n}\\
				\hline 
				\SI{23,5}{} & \SI{-9,88}{} & \SI{5,13e-5}{}   & \SI{3,37e-3}{}	& \SI{2,21}{} \\
				\SI{30,2}{} & \SI{-9,38}{} & \SI{8,48e-5}{}   & \SI{3,30e-3}{} 	& \SI{2,19}{}\\
				\SI{37,0}{} & \SI{-8,83}{} & \SI{1,47e-4}{}   &   \SI{3,22e-3}{}& \SI{2,14}{}  \\
				\hline	
\end{tabular}}}
\end{table}
\FloatBarrier
\vspace*{-2.5mm}
%Tabelle Ende

Für die Berechnung des Stoßfaktors $k_0$ und der Aktivierungsenergie $E_A$ können die berechneten kinetischen Kennwerte Stoßfaktor $k$ und die reziproke Temperatur $\frac{1}{T}$ in einem \textsc{Arrhenius}-Diagramm dargestellt werden (siehe Abb. \ref{dia:lnk/(1/T)}).

\begin{figure}[h!]
	\begin{center}
		%\resizebox{0.8\textwidth}{!}{
		\begin{tikzpicture}[trim axis left, trim axis right]
		\begin{axis}[
		axis lines = left,
		width = 13cm,
		height = 7cm,
		xmin = 0.0032,
		xmax = 0.0034,
		ymin = -10,
		ymax = -8.4,
		ytick = {-10,-9.75,...,-8.5},
		xtick = {0.0032,0.00325,...,0.0034},
		ylabel={ln(k)},
		y label style={at={(-0.03,0.5)}},
		xlabel={1/T in [K]},
		legend style={at={(0.1,0.95)},anchor=west}
		]
		\addplot table {Arrhenius1.dat};
		\addplot +[mark=none, dashed, black,domain=0.0032:0.0034] {-7170.5*x + 14.283};
		\legend{\text{$y(x) = -7170,5*x + 14,283 \quad | \quad R^2 = 0,9989$}};
		\end{axis}
		\end{tikzpicture}
		%	}Ventilkennlinie
		\caption{\textsc{Arrhenius}-Diagramm}
		\label{dia:lnk/(1/T)}
	\end{center}
\end{figure}
\FloatBarrier
\vspace*{-15mm}

\begin{flalign}
	\ln k &= \ln k_0 - \frac{E_A}{R*T}\\
	E_A	&= (\ln k_0-\ln k)*R*T
\end{flalign}
\vspace*{-5mm}

%Tabelle START
\renewcommand{\arraystretch}{1.2}
\begin{table}[h!]
	\centering
	\caption{Kinetische Konstanten}
	\label{tab:kin_kon}
	\makebox[\textwidth]{
		\resizebox{17cm}{!}{
			\begin{tabular}{c|c|c}
				\hline
				\textbf{$\ln k_0$}&\textbf{Stoßfaktor $k_0$ in \si{\liter\per\milli \mole \per \second}} & \textbf{Aktivierungsenergie $E_A$ in \si{\kilo \joule \per \mole}} \\
				\hline
				\SI{14,283}{}&\SI{1,6e6} & \SI{59,6}{}\\				
				\hline	
	\end{tabular}}}
\end{table}
\FloatBarrier
\vspace*{-2.5mm}
%Tabelle Ende

\subsection{Verweilzeit}
\label{subsec:verweilzeit}

\newcommand*\diff{\mathop{}\!\mathrm{d}}
\newcommand*\Diff[1]{\mathop{}\!\mathrm{d^#1}}

\textbf{Berechnung von E(t) bei der Impulsmarkierung:}
\begin{flalign}
	E(t_m) &= \frac{c(t_m)}{\int_{0}^{\infty} c(t) \diff{t}} \approx \frac{c(t_m)}{\left(\frac{c_0}{2}+c_1+...+c_{L-1}+\frac{c_L}{2}\right)*\Delta t}
\end{flalign}

\textbf{Berechnung von F(t) bei der Impulsmarkierung:}
\begin{flalign}
F(t_m) &= \int_{t_0}^{t_m} E(t) \diff{t} \approx \left(\frac{E(t)_0}{2}+c_1+...+c_{m-1}+\frac{c_m}{2}\right)*\Delta t
\end{flalign}

\begin{figure}[h!]
	\begin{center}
		%\resizebox{0.8\textwidth}{!}{
		\begin{tikzpicture}[trim axis left, trim axis right]
		\begin{axis}[
		axis lines = left,
		width = 16cm,
		height = 8cm,
		xmin = 0,
		xmax = 2500,
		ymin = 0,
		ymax = 0.0015,
		ylabel={E(t) in \si{\per \second}},
		y label style={at={(-0.03,0.5)}},
		xlabel={t in \si{\second}},
		legend style={at={(0.95,0.95)},anchor=west}
		]
		\addplot table {E(t)_impuls_1.dat};
		\addplot table {E(t)_impuls_2.dat};
		\addplot table {E(t)_impuls_3.dat};
		\legend{RK1,RK1+2,RK1+2+3};
		\end{axis}
		\end{tikzpicture}
		%	}Ventilkennlinie
		\caption{E(t) Impulsmarkierung}
		\label{dia:E(t) impuls}
	\end{center}
\end{figure}
\FloatBarrier

\begin{figure}[h!]
	\begin{center}
		\resizebox{1.0\textwidth}{!}{
		\begin{tikzpicture}[trim axis left, trim axis right]
		\begin{axis}[
		axis lines = left,
		width = 16cm,
		height = 8cm,
		xmin = 0,
		xmax = 2500,
		ymin = 0,
		ymax = 1,
		ylabel={F(t)},
		y label style={at={(-0.03,0.5)}},
		xlabel={t in \si{\second}},
		legend style={at={(0.75,0.75)},anchor=west}
		]
		\addplot table {F(t)_impuls_1.dat};
		\addplot table {F(t)_impuls_2.dat};
		\addplot table {F(t)_impuls_3.dat};
		\legend{RK1,RK1+2,RK1+2+3};
		\end{axis}
		\end{tikzpicture}
			}
		\caption{F(t) Impulsmarkierung}
		\label{dia:F(t) Impuls}
	\end{center}
\end{figure}
\FloatBarrier

\textbf{Berechnung von F(t) bei der Verdrängungsmarkierung:}
\begin{flalign}
F(t_m) &= \frac{c(t_m)}{c_0}
\end{flalign}

\textbf{Berechnung von E(t) bei der Verdrängungsmarkierung:}
\begin{flalign}
E(t_m) &= \frac{\diff F(t_m)}{\diff t} \approx \frac{F(t_{m+1})-F(t_{m-1})}{t_{m+1}-t_{m-1}} \approx \frac{F(t_{m+1})-F(t_{m-1})}{2*\Delta t}
\end{flalign}

\newpage

\textbf{Berechnung der Verweilzeit $\bar{t}$ bzw. $\tau$}
\begin{flalign}
\tau &\approx \left[\frac{t_0*E_0}{2}+t_1*E_1+...+t_{L-1}*E_{L-1}+\frac{t_L*E_L}{2}\right]*\Delta t
\end{flalign}
\textbf{Berechnung der Varianz $\sigma^2$}
\begin{flalign}
\sigma^2 &\approx \left[\frac{t_0^2*E_0}{2}+t_1^2*E_1+...+t_{L-1}^2*E_{L-1}+\frac{t_L^2*E_L}{2}\right]*\Delta t-(\tau)^2
\end{flalign}
\textbf{Berechnung der genormten Varianz $\sigma_\Theta^2$}
\begin{flalign}
\sigma_\Theta^2 &= \frac{\sigma^2}{\tau^2}
\end{flalign}
\textbf{Berechnung der Bodensteinzahl $Bo$}\\
Startwert: $Bo = \frac{2}{\sigma_\Theta^2}$
\begin{flalign}
	Bo &= \frac{2}{\sigma_\Theta^2+\frac{2}{Bo^2}*\left(1- \mathrm{e}^{-Bo}\right)}
\end{flalign}
\textit{{\small Iterativ ermittelt über Excel VBA Makro mit einer Abweichung von $10^{-5}$ zur vorherigen Bodensteinzahl.}}\\

\textbf{Berechnung der Kesselzahl $K$}\\
\begin{flalign}
	K=1+\frac{1}{2}*\sqrt{{Bo}^2+1}
\end{flalign}

\begin{figure}[h!]
	\begin{center}
		%\resizebox{0.8\textwidth}{!}{
		\begin{tikzpicture}[trim axis left, trim axis right]
		\begin{axis}[
		axis lines = left,
		width = 16cm,
		height = 8cm,
		xmin = 0,
		xmax = 2500,
		ymin = 0,
		ymax = 1,
		ylabel={F(t)},
		y label style={at={(-0.03,0.5)}},
		xlabel={t in \si{\second}},
		legend style={at={(0.75,0.75)},anchor=west}
		]
		\addplot table {F(t)_verdrang_1.dat};
		\addplot table {F(t)_verdrang_2.dat};
		\addplot table {F(t)_verdrang_3.dat};
		\legend{RK1,RK1+2,RK1+2+3};
		\end{axis}
		\end{tikzpicture}
		%	}Ventilkennlinie
		\caption{F(t) Verdrängung}
		\label{dia:F(t) verdrang}
	\end{center}
\end{figure}
\FloatBarrier

\begin{figure}[h!]
	\begin{center}
		\resizebox{1.0\textwidth}{!}{
		\begin{tikzpicture}[trim axis left, trim axis right]
		\begin{axis}[
		axis lines = left,
		width = 16cm,
		height = 8cm,
		xmin = 0,
		xmax = 2500,
		ymin = 0,
		ymax = 0.0035,
		ylabel={E(t)},
		y label style={at={(-0.03,0.5)}},
		xlabel={t in \si{\second}},
		legend style={at={(0.75,0.75)},anchor=west}
		]
		\addplot table {E(t)_verdrang_1.dat};
		\addplot table {E(t)_verdrang_2.dat};
		\addplot table {E(t)_verdrang_3.dat};
		\legend{RK1,RK1+2,RK1+2+3};
		\end{axis}
		\end{tikzpicture}
			}
		\caption{E(t) Verdrängung}
		\label{dia:E(t) verdrang}
	\end{center}
\end{figure}
\FloatBarrier

%Tabelle START
\renewcommand{\arraystretch}{1.2}
\begin{table}[h!]
	\centering
	\caption{Berechnete Werte aus den Verweilzeiten der Impulsmarkierung}
	\label{tab:vwz_impuls}
	\makebox[\textwidth]{
		\resizebox{15cm}{!}{
			\begin{tabular}{c|c|c|c}
				\hline
				 & \textbf{RK1}& \textbf{RK1+RK2} &  \textbf{RK1+RK2+RK3} \\
				\hline
				\textbf{VWZ in \si{\second}} & 358 & 605 & 944 \\
				\hline
				\hline
				\textbf{Varianz in \si{\raiseto{2} \second}} & \SI{134459}{}   & \SI{128017}{} & \SI{177191}{} \\
				\hline
				\hline
				\textbf{Normierte Varianz} & 1,05 & 0,35 & 0,20 \\
				\hline
				\hline
				\textbf{Bodensteinzahl} & \SI{1,9e-2} & 4,5 & 8,9 \\
				\hline	
				\hline
				\textbf{Kesselzahl} & 1,5 & 3,3 & 5,5 \\
				\hline		
	\end{tabular}}}
\end{table}
\FloatBarrier
\vspace*{-2.5mm}
%Tabelle Ende

%Tabelle START
\renewcommand{\arraystretch}{1.2}
\begin{table}[h!]
	\centering
	\caption{Berechnete Werte aus den Verweilzeiten der Verdrängungsmarkierung}
	\label{tab:vwz_verdrang}
	\makebox[\textwidth]{
		\resizebox{15cm}{!}{
			\begin{tabular}{c|c|c|c}
				\hline
				& \textbf{RK1}& \textbf{RK1+RK2} &  \textbf{RK1+RK2+RK3} \\
				\hline
				\textbf{VWZ in \si{\second}} & 394 & 951 & 1529 \\
				\hline
				\hline
				\textbf{Varianz in \si{\raiseto{2} \second}} & \SI{75059}{}   & \SI{372779}{} & \SI{1165252}{} \\
				\hline
				\hline
				\textbf{Normierte Varianz} & 0,48 & 0,41 & 0,50 \\
				\hline	
				\hline
				\textbf{Bodensteinzahl} & 2,7 & 3,5 & 2,6 \\
				\hline	
				\hline
				\textbf{Kesselzahl} & 2,5 & 2,8 & 2,4 \\
				\hline		
	\end{tabular}}}
\end{table}
\FloatBarrier
\vspace*{-2.5mm}
%Tabelle Ende

\subsection{Umsatzgrad}
\label{subsec: umsatzgrad}

\textbf{Berechnung der experimentell bestimmten Umsatzgrade:}
\begin{flalign}
%\label{gl:}
X_{\text{experimentell}} &= \frac{|\lambda_0-\lambda|}{|\lambda_0-\lambda_\infty|}
\end{flalign}
\vspace*{-10mm}

%Tabelle START
\renewcommand{\arraystretch}{1.2}
\begin{table}[h!]
	\centering
	\caption{Experimentell bestimmte Umsatzgrade}
	\label{tab:usg_experimentell}
	\makebox[\textwidth]{
		\resizebox{11cm}{!}{
			\begin{tabular}{c|c|c|c}
				\hline
				& \textbf{RK1}& \textbf{RK1+RK2} &  \textbf{RK1+RK2+RK3} \\
				\hline
				\textbf{X in \si{\percent}} & $\approx \SI{49}{}$ & $\approx \SI{66}{}$ & $\approx \SI{73}{}$ \\
				\hline	
	\end{tabular}}}
\end{table}
\FloatBarrier
\vspace*{-2.5mm}
%Tabelle Ende

\textbf{Berechnung der analytisch bestimmten Umsatzgrade aus den kinetischen Kennwerten:}
\begin{flalign}
%\label{}
X_{\text{analytisch}} &= 1-\left[1+k(T)*(n-1)*c_{A,0}^{n-1}*t\right]^{\frac{1}{1-n}}
\end{flalign}
\vspace*{-10mm}


%Tabelle START
\renewcommand{\arraystretch}{1.2}
\begin{table}[h!]
	\centering
	\caption{Analytische Umsatzgrade aus den kinetischen Kennwerten}
	\label{tab:usg_analytisch}
	\makebox[\textwidth]{
		\resizebox{10cm}{!}{
			\begin{tabular}{c|c|c|c}
				\hline
				& \textbf{T=\SI{23,5}{\celsius}}& \textbf{T=\SI{30,2}{\celsius}} &  \textbf{T=\SI{37,0}{\celsius}} \\
				\hline
				\textbf{X in \si{\percent}} & $\approx \SI{64}{}$ & $\approx \SI{72}{}$ & $\approx \SI{77}{}$ \\
				\hline	
	\end{tabular}}}
\end{table}
\FloatBarrier
%Tabelle Ende

\textbf{Berechnung der mittleren Umsatzgrade aus E(t) der Impulsmarkierung und $X(t)_{\text{analytisch}}$ (Inkrementmethode)}
\begin{flalign}
%\label{}
	\bar{X} &= \int_{0}^{\infty} E(t)*X_{\text{analytisch}}(t) \diff t \approx (E_1*X_1+...+E_L*X_L)*\Delta t
\end{flalign}
\vspace*{-10mm}

%Tabelle START
\renewcommand{\arraystretch}{1.2}
\begin{table}[h!]
	\centering
	\caption{Mittlere Umsatzgrade aus E(t) der Impulsmarkierung und $X(t)_{\text{analytisch}}$ (Inkrementmethode)}
	\label{tab:usg_real}
	\makebox[\textwidth]{
		\resizebox{13cm}{!}{
			\begin{tabular}{c|c|c|c}
				\hline
				& \textbf{RK1}& \textbf{RK1+RK2} &  \textbf{RK1+RK2+RK3} \\
				\hline
				\textbf{X in \si{\percent} bei T= \SI{23,5}{\celsius}} & $\approx \SI{52}{}$ & $\approx \SI{69}{}$ & $\approx \SI{79}{}$\\
				\hline	
				\textbf{X in \si{\percent} bei T= \SI{30,2}{\celsius}} & $\approx \SI{60}{}$ & $\approx \SI{77}{}$ & $\approx \SI{89}{}$\\
				\hline	
				\textbf{X in \si{\percent} bei T= \SI{37,0}{\celsius}} & $\approx \SI{67}{}$ & $\approx \SI{82}{}$ & $\approx \SI{89}{}$\\
				\hline		
	\end{tabular}}}
\end{table}
\FloatBarrier
%Tabelle Ende

\textbf{Bestimmung der idealen Umsatzgrade aus E(t) der Impulsmarkierung und $X(t)_{\text{analytisch}}$ (Solver):}
\begin{flalign}
	r_a &= X*\frac{c_0}{\tau} \\
	r_b&= k*c_{A,0}^n*(1-X)^n\\
	r_a&=r_b\\
	X*\frac{c_0}{\tau} &= k*c_{A,0}^n*(1-X)^n
\end{flalign}
\textit{Methode des kleinsten Quadrates: gesucht ist Min$[\bar{x}^2=(|ra-rb|)^2]$}


%Tabelle START
\renewcommand{\arraystretch}{1.2}
\begin{table}[h!]
	\centering
	\caption{Ideale Umsatzgrade aus E(t) der Impulsmarkierung und $X(t)_{\text{analytisch}}$ (Solver)}
	\label{tab:usg_ideal}
	\makebox[\textwidth]{
		\resizebox{13cm}{!}{
			\begin{tabular}{c|c|c|c}
				\hline
				& \textbf{RK1}& \textbf{RK1+RK2} &  \textbf{RK1+RK2+RK3} \\
				\hline
				\textbf{X in \si{\percent} bei T= \SI{23,5}{\celsius}} & $\approx \SI{50}{}$ & $\approx \SI{71}{}$ & $\approx \SI{80}{}$\\
				\hline	
				\textbf{X in \si{\percent} bei T= \SI{30,2}{\celsius}} & $\approx \SI{57}{}$ & $\approx \SI{75}{}$ & $\approx \SI{82}{}$\\
				\hline	
				\textbf{X in \si{\percent} bei T= \SI{37,0}{\celsius}} & $\approx \SI{62}{}$ & $\approx \SI{78}{}$ & $\approx \SI{84}{}$\\
				\hline		
	\end{tabular}}}
\end{table}
\FloatBarrier
\vspace*{-2.5mm}
%Tabelle Ende


\begin{figure}[h!]
	\begin{center}
		\resizebox{0.90\textwidth}{!}{
			\begin{tikzpicture}[trim axis left, trim axis right]
			\begin{axis}[
			axis lines = left,
			width = 16cm,
			height = 6.5cm,
			xmin = 0.4,
			xmax = 1.,
			ymin = 0,
			ymax = 0.12,
			ylabel={r(T=\SI{23,5}{\celsius})},
			y label style={at={(-0.03,0.5)}},
			xlabel={Umsatzgrad X},
			legend style={at={(0.85,0.75)},anchor=west}
			]
			\addplot table {r20.dat};
			\addplot table {r rk1.dat};
			\addplot table {x20 rk1 rk2.dat};
			\addplot table {x20 rk1 rk2 rk3.dat};
			\addplot +[mark=none, dashed, black] coordinates {(0.497491049, 0) (0.497491049, 0.25)};
			\addplot +[mark=none, dashed, black] coordinates {(0.708500151, 0) (0.708500151, 0.25)};
			\addplot +[mark=none, dashed, black] coordinates {(0.801926384, 0) (0.801926384, 0.25)};
			\legend{RK1,RK2+RK2,RK1+RK2+RK3};
			\end{axis}
			\end{tikzpicture}
		}
		\caption{Grafische Bestimmung der idealen Umsatzgrade bei T=\SI{23,5}{\celsius}}
		\label{dia:r20}
	\end{center}
\end{figure}
\FloatBarrier
\vspace*{-10mm}

\begin{figure}[h!]
	\begin{center}
		\resizebox{0.90\textwidth}{!}{
			\begin{tikzpicture}[trim axis left, trim axis right]
			\begin{axis}[
			axis lines = left,
			width = 16cm,
			height = 6.5cm,
			xmin = 0.4,
			xmax = 1.,
			ymin = 0,
			ymax = 0.2,
			ylabel={r(T=\SI{30,2}{\celsius})},
			y label style={at={(-0.03,0.5)}},
			xlabel={Umsatzgrad X},
			legend style={at={(0.85,0.75)},anchor=west}
			]
			\addplot table {r30.dat};
			\addplot table {r rk1.dat};
			\addplot table {x30 rk1 rk2.dat};
			\addplot table {x30 rk1 rk2 rk3.dat};
			\addplot +[mark=none, dashed, black] coordinates {(0.566893083, 0) (0.566893083, 0.25)};
			\addplot +[mark=none, dashed, black] coordinates {(0.745913745, 0) (0.745913745, 0.25)};
			\addplot +[mark=none, dashed, black] coordinates {(0.823956571, 0) (0.823956571, 0.25)};
			\legend{RK1,RK2+RK2,RK1+RK2+RK3};
			\end{axis}
			\end{tikzpicture}
		}
		\caption{Grafische Bestimmung der idealen Umsatzgrade bei T=\SI{30,2}{\celsius}}
		\label{dia:r30}
	\end{center}
\end{figure}
\FloatBarrier
\vspace*{-5mm}

\begin{figure}[h!]
	\begin{center}
		\resizebox{0.90\textwidth}{!}{
			\begin{tikzpicture}[trim axis left, trim axis right]
			\begin{axis}[
			axis lines = left,
			width = 16cm,
			height = 6.5cm,
			xmin = 0.4,
			xmax = 1.,
			ymin = 0,
			ymax = 0.25,
			ylabel={r(T=\SI{37,0}{\celsius})},
			y label style={at={(-0.03,0.5)}},
			xlabel={Umsatzgrad X},
			legend style={at={(0.85,0.75)},anchor=west}
			]
			\addplot table {r40.dat};
			\addplot table {r rk1.dat};
			\addplot table {x40 rk1 rk2.dat};
			\addplot table {x40 rk1 rk2 rk3.dat};
			\addplot +[mark=none, dashed, black] coordinates {(0.620463586, 0) (0.620463586, 0.25)};
			\addplot +[mark=none, dashed, black] coordinates {(0.775, 0) (0.775, 0.25)};
			\addplot +[mark=none, dashed, black] coordinates {(0.84426, 0) (0.84426, 0.25)};
			\legend{RK1,RK2+RK2,RK1+RK2+RK3};
			\end{axis}
			\end{tikzpicture}
		}
		\caption{Grafische Bestimmung der idealen Umsatzgrade bei T=\SI{37,0}{\celsius}}
		\label{dia:r40}
	\end{center}
\end{figure}
\FloatBarrier
\vspace*{-2.5mm}





\subsection{Berechnung eines kontinuierlichen Reaktors}
\begin{itemize}
	\item \SI{100 000}{\tonne\per\year} Natriumacetat
	\item \SI{8000}{\tonne\per\year}
	\item Reaktionstemperatur \SI{32}{\degreeCelsius}
	\item Umsatzgrad $X=$ 80\%
\end{itemize}



\begin{flalign}
\dot{m}=	\frac{\SI{100 000}{\tonne\per\year}}{\SI{8000}{\tonne\per\year}}&= \SI{12,5}{\tonne\per\hour} =3472,2\si{\gram\per\second}\\
\dot{n}=\frac{\dot{m}}{M} = \frac{\SI{3472,2}{\gram\per\second}}{\SI{82,03}{\gram\per\mole}}&=
 \SI{42,34}{\mole\per\second}\\
\end{flalign}

\begin{flalign}
	\dot{n}_\omega(NaAc)&=	\dot{n}_\alpha(NaOH)*X\\
	\dot{n}_\alpha(NaOH)&= \frac{\dot{n}_\omega(NaAc)}{X}\\
	&=\frac{\SI{42,34}{\mole\per\second}}{0,8}= \SI{52,93}{\mole\per\second}
\end{flalign}

\begin{flalign}
	k_{32\si{\degreeCelsius}}&=k_0*e^{-\frac{E_A}{R*T}}\\
	&=1,6*10^6*e^{-\frac{59,6\si{\joule\per\mole}*10^3}{\SI{8,314}{\joule\per\mole\per\kelvin}*\SI{305}{\kelvin}}}*\si{\liter \per \milli \mol \per \second}\\
	&=9,96*10^{-5}\si{\liter\per\milli\mole\per\second}
\end{flalign}
\textit{Im Folgenden wird die Reaktionsordnung n = 2  und eine Verweilzeit von $\tau=\SI{0,25}{\hour}$ angenommen. Als Reaktor wurde sich auf einen kontinuierlichen Rührkessel festgelegt.}
\begin{flalign}
	X*\frac{c_0}{\tau}&=k*c_0^n*(1-X)^n\\
	c_0&=\frac{X}{\tau*k*(1-X)^n}\\
	&= \frac{0,8}{\SI{900}{\second}*9,96*10^{-5}\si{\cubic\meter\per\mole\per\second}*(1-0,8)^2}\\
	&=223,115\si{\mole\per\cubic\meter}
	\end{flalign}
	
	\begin{flalign}
		\dot{V}&=\frac{\dot{n}}{c_0}
		=\frac{\SI{53}{\mole\per\second}}{\SI{0,223}{\mole\per\liter}}\\
		&=237,5\si{\liter\per\second}
	\end{flalign}
	
	\begin{flalign}
		\tau&=\frac{V_R}{\dot{V}}\\
		V_R&=\tau*\dot{V}=0,2375\si{\cubic\meter\per\second}*900\si{\second}\\
		&\approx \underline{\underline{215 \si{\cubic\meter}}}
	\end{flalign}