\section{Auswertung}
\label{sec:auswertung}

%Darstellung Messungen und Kurzbeschreibung

\begin{figure}[h!]
	\begin{center}
		%\resizebox{0.8\textwidth}{!}{
		\begin{tikzpicture}[trim axis left, trim axis right]
		\begin{axis}[
		axis lines = left,
		width = 13cm,
		height = 7cm,
		xmin = 0,
		xmax = 20,
		ymin = 0,
		ymax = 0.1,
		%ytick = {0,2,...,14},
		%xtick = {0,10,...,100},
		ylabel={c in [\si{\mole\per\liter}]},
		xlabel={t in [\si{\second}]},
	%	legend style={at={(0.5,0.95)},anchor=west}
		]
		\addplot table {rt-M-x-replace-string.dat};
		\end{axis}
		\end{tikzpicture}
		%	}Ventilkennlinie
		\caption{Konzentration über der Zeit}
		\label{dia:c/t}
	\end{center}
\end{figure}
\FloatBarrier

\begin{figure}[h!]
	\begin{center}
		%\resizebox{0.8\textwidth}{!}{
		\begin{tikzpicture}[trim axis left, trim axis right]
		\begin{axis}[
		axis lines = left,
		width = 13cm,
		height = 7cm,
		xmin = 0,
		xmax = 20,
		ymin = 0,
		ymax = 0.1,
		%ytick = {0,2,...,14},
		%xtick = {0,10,...,100},
		ylabel={ln(r) in []},
		xlabel={ln(c) in []},
		%	legend style={at={(0.5,0.95)},anchor=west}
		]
		\addplot table {rt-M-x-replace-string.dat};
		\end{axis}
		\end{tikzpicture}
		%	}Ventilkennlinie
		\caption{Konzentration über der Zeit}
		\label{dia:lnr/lnc}
	\end{center}
\end{figure}
\FloatBarrier

\begin{figure}[h!]
	\begin{center}
		%\resizebox{0.8\textwidth}{!}{
		\begin{tikzpicture}[trim axis left, trim axis right]
		\begin{axis}[
		axis lines = left,
		width = 13cm,
		height = 7cm,
		xmin = 0.0032,
		xmax = 0.0034,
		ymin = -10,
		ymax = -8.4,
		ytick = {-10,-9.75,...,-8.5},
		xtick = {0.0032,0.00325,...,0.0034},
		ylabel={ln(k)},
		y label style={at={(-0.03,0.5)}},
		xlabel={1/T in [K]},
		%	legend style={at={(0.5,0.95)},anchor=west}
		]
		\addplot table {Arrhenius1.dat};
		\end{axis}
		\end{tikzpicture}
		%	}Ventilkennlinie
		\caption{Arrhenius-Diagramm}
		\label{dia:lnk/(1/T)}
	\end{center}
\end{figure}
\FloatBarrier

\subsection{Berechnung eines kontinuierlichen Reaktors}
\begin{itemize}
	\item \SI{100 000}{\tonne\per\year} Natriumacetat
	\item Reaktionstemperatur \SI{32}{\degreeCelsius}
	\item Umsatzgrad $X=$ 80\%
\end{itemize}
