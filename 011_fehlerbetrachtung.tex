\section{Fehlerbetrachtung}
\label{sec:fehler}

Bei der Herstellung der Natronlauge für den Versuchsteil Kinetik ging etwas Lösung verloren. Dies wurde dadurch ersichtlich, dass beim befüllen von drei Maßkolben zu je \SI{1}{\litre}, ein Maßkolben nicht gänzlich gefüllt werden konnte. Der Verlust wird auf maximal \SI{5}{\milli\liter} geschätzt. Die resultierende Abweichung beträgt damit 0,5\% und kann gegenüber anderen zufälligen Fehlern vernachlässigt werden. 

Im Versuchsteil Umsatzgrad wurde der Verlauf dadurch gestört, dass eine Schlauchpumpe Luft zog. Die Reaktionslösung setzte sich damit ab einem schwer zu bestimmenden Zeitpunkt nicht mehr äquimolar zusammen. Die Endleitfähigkeit konnte daher nicht aus dem Produktgemisch ermittelt werde. Anstatt dieser wurde für die Berechnungen die mittlere Endleitfähigkeit aus dem Versuchsteil Kinetik noch einmal verwendet. 

Der Essigsäureethylether besitzt einen Flammpunkt von \SI{-4}{\degreeCelsius} und ist damit als leicht flüchtig einzustufen. Während der Umsetzung im offenen Rührkessel, insbesondere bei Temperaturen bis knapp \SI{40}{\degreeCelsius}, könnte ein geringer Anteil verdampft sein. Da die zugegebene Menge auf $\frac{1}{100}$\si{\milli\liter} genau abgemessen wurde, könnte sich dieser Einfluss auf die Reaktion ausgewirkt haben.


Allgemein sind Mess- und Ableseungenauigkeiten in vernachlässigbarer Größenordnung anzunehmen.    