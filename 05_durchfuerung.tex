\section{Durchführung}
\label{sec:durchfuerung}

\subsection{Durchführung zum Versuch Kinetik}
Der Verwendete Reaktor wies ein Volumen von \SI{1}{\liter} auf. Das Reaktionsgemisch wurde mit einem Ankerrührer bei einer Drehzahl von \SI{100}{\per\minute} homogenisiert. Ein Thermostat sorgte für konstante Temperaturen. 
Der Versuch wurde drei mal wiederholt. Dabei lagen die Temperaturen bei \SI{23,5}{\degreeCelsius}, \SI{30,2}{\degreeCelsius} und \SI{37,0}{\degreeCelsius}.

Es wurde die hydrolytische Spaltung des Essigsäureethylesters in alkalischem Millieu untersucht. 

Einer Vorlage aus \SI{1}{\liter} 0,05 molarer Natriumhydroxidlösung wurde zum Zeitpunkt Null jeweils 0,05 Mol reiner Essigsäureethylether zugegeben.   

Die Leitfähigkeitsmessung erfolgte automatisch alle 5 Sekunden durch das \textsc{inoLab-IDS Multi\nolinebreak 9430} in Verbindung mit einer entsprechenden Sonde automatisch für annähernd 15 Minuten.

\subsection{Durchführung zum Versuch Verweilzeit}

Anstelle des einfachen Rührkessels wurde hier eine Rührkesselkaskade aus drei Rührkesseln eingesetzt. Jeder Kessel verfügte über einen eigenen Rührer. Es wurde Deionisiertes Wasser und eine eingefärbte, schwache Salzlösung verwendet. 
Im ersten Teilversuch wurde durch die Zugabe einer geringen Menge Salzlösung eine Impulsmarkierung erzeugt.  
Der Volumenstrom an deionisiertem Wasser belief sich auf \SI{8}{\liter\per\hour}.\linebreak
Im zweiten Teil fand eine Verdrängung des deionisierten Wassers durch das leicht salzbeladene Wasser statt. Der Salzwasservolumenstrom betrug in diesem Fall \SI{10}{\liter\per\hour}.

Messstellen für die Leitfähigkeit befanden sich hinter jedem Kessel. \\Die Leitfähigkeitswerte wurden automatisch alle fünf Sekunden über den Computer erfasst.
Zur Dosierung des Wassers wurde über Schlauchpumpen verwirklicht.
\subsection{Durchführung zum Versuch Umsatzgrad}

In Vorbereitung auf den Versuch zum Umsatzgrad wurden je \SI{6}{\liter} 0,1 molare Natriumhydroxid- und Essigsäureethylesterlösung angemischt. Die Anlage mit der Rührkesselkaskade wurde entleert. Es wurde der gleiche Gesamtvolumenstrom wie im zweiten Teil des Versuches zur Verweilzeit, von \SI{10}{\liter\per\hour} eingestellt. Dieser setzte sich aus \SI{5}{\liter\per\hour} Natriumhydroxidlösung und \SI{5}{\liter\per\hour} Essigsäuereethylesterlösung zusammen. Die Leitfähigkeitswerte wurden nach jedem der drei Kessel automatisch alle fünf Sekunden aufgenommen.




